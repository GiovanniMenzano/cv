%!TEX TS-program = xelatex
%!TEX encoding = UTF-8 Unicode

%%% CONFIGURATION %%%
\documentclass[]{style}
\usepackage{textcomp}
%%% Override a directory location for fonts(default: 'fonts/')
\fontdir[fonts/]

%%% Configure a directory location for sections
\newcommand*{\sectiondir}{resume/}

%%% Override color
% Awesome Colors: awesome-emerald, awesome-skyblue, awesome-red, awesome-pink, awesome-orange, awesome-nephritis, awesome-concrete, awesome-darknight
%% Color for highlight
% Define your custom color if you don't like awesome colors
\colorlet{awesome}{awesome-red}
%\definecolor{awesome}{HTML}{242424}
%% Colors for text
%\definecolor{darktext}{HTML}{414141}
%\definecolor{text}{HTML}{414141}
%\definecolor{graytext}{HTML}{414141}
%\definecolor{lighttext}{HTML}{414141}

%%% Override a separator for social informations in header(default: ' | ')
%\headersocialsep[\quad\textbar\quad]
    \begin{document}

%%% PROFILE %%%
\begin{center}
	\headerfirstnamestyle{Giovanni} \headerlastnamestyle{Menzano} \\
	\vspace{2mm}
	\headerinfostyle{{\faEnvelope\space\space GIOVANNI.MENZANO@GMAIL.COM}\space\space\space |\space\space\space {\faMapMarker\space CROTONE, KR}}
\end{center}

%%% EXPERIENCE %%%
\cvsection{Esperienza lavorativa}
\begin{cventries}
	\cventry
	{Sviluppatore software}
	{IBM}
	{Crotone, KR}
	{Ottobre 2021 – presente}
	{\begin{cventryparagraph}
		\item {Progetto Istat AIC - Applicazioni in cloud.}
		\item {Gestione e sviluppo di un pool di 40 applicazioni utilizzate da Istat, dai comuni e dalle aziende italiane per numerose attività, quali: pubblicare dati statistici, analizzare e organizzare informazioni, compilare questionari e molto altro. Gli incarichi possono essere suddivisi in 3 categorie:}
    	{\begin{cventryparagraphlist}
    		\item {Risolvere gli incidenti di produzione segnalati dai dipendenti Istat;}
    		\item {organizzare e completare le attività applicative predeterminate durante l'anno, come il caricamento dei dati, l'autorizzazione degli utenti a determinate applicazioni e funzioni, la gestione del database e del server;}
    		\item {richieste di modifica e correzioni.}
    	\end{cventryparagraphlist}}
	\end{cventryparagraph}}
	\cventry
	{Sviluppatore software}
	{NTT Data}
	{Rende, CS}
	{Aprile 2020 – Settembre 2021}
	{\begin{cventryparagraph}
		\item {Progetto Enel-x DBE (Digital Backend).}
		\item {Software scritto in Java, utilizzando il framework Spring, servizi REST e architettura a microservizi. Persistenza dei dati implementata con Hibernate su database PostgreSQL. Impiegato come analista per le attività di AMS, che comprendono analisi e correzione di bug, richieste dal backoffice e gestione degli incidenti di produzione. Principali responsabilità:}
    	{\begin{cventryparagraphlist}
    		\item {analisi dei log tramite servizio Cloudwatch di AWS}
    		\item {estrazione dati and manutenzione del database;}
    		\item {servizio di reperibilità per richieste di monitoraggio, controllo dello stato di salute dei microservizi e riavvio in caso di necessità;}
    		\item {comunicazione e gestione del cliente;}
    		\item {gestione delle attività del team AMS e affiacamento di nuovi colleghi nella stessa posizione.}
    	\end{cventryparagraphlist}}
	\end{cventryparagraph}}
	\cventry
	{Sviluppatore software}
	{Exprivia}
	{Rende, CS}
	{Settembre 2019 – Marzo 2020}
	{\begin{cventryparagraph}
		\item {Progetto SISR (Sistema Informativo Sanitario Regionale).}
		\item {Sviluppato in Java, con l'uso di tecnologia EJB, su database Oracle. Il progetto è inizialmente partito per la regione Puglia, il nostro obiettivo è quello di esportare e adattare il codice in base alle esigenze della regione Calabria. Le attività comprendono:}
    	{\begin{cventryparagraphlist}
    		\item {porting e adattamento di funzioni e librerie utilizzate;}
    		\item {porting e test dei ruoli Oracle e altri dati di configurazione;}
    		\item {sviluppo di nuove funzionalità per necessità della regione.}
    	\end{cventryparagraphlist}}
	\end{cventryparagraph}}
	\cventry
	{Sviluppatore full stack}
	{Consoft/Reply}
	{Milano, MI}
	{Dicembre 2018 – Agosto 2019}
	{\begin{cventryparagraph}
		\item {Progetto UnipolSai Contest.}
		\item {Applicazione web utilizzata per la visualizzazione del progresso di competizioni e premi produzione per le filiali UnipolSai. Progetto sviluppato in Java e Spring, utilizzando le api REST. Frontend sviluppato in HTML/CSS/JS con l'aiuto del toolkit Dojo. Relativamente ai dati, il software utilizza Mybatis e un database DB2.
		Il team è composto da me, che come sviluppatore mi occupo di frontend e backend, e dal team leader responsabile fondamentalmente della gestione del progetto. Principali attività:}
    	{\begin{cventryparagraphlist}
    		\item {costruzione delle query per il recupero dei dati da database;}
    		\item {sviluppo e aggiornamento di servizi REST;}
    		\item {sviluppo e aggiornamento dei widget Dojo usati per visualizzare grafici e classifiche;}
    		\item {creazione delle pagine su Websphere Portal;}
    		\item {esecuzione di test e mock di dati;}
    		\item {rilasci su ambienti di sviluppo e collaudo.}
    	\end{cventryparagraphlist}}
	\end{cventryparagraph}}
\end{cventries}
\pagebreak[4]

%%% EDUCATION %%%
\cvsection{Istruzione}
\begin{cventries}
	\cventry
	{Ingegneria informatica}
	{Università della Calabria}
	{Rende, CS}
	{2011 – in sospeso}
	{}
	\cventry
	{Scienze}
	{Liceo Scientifico Filolao}
	{Crotone, KR}
	{2006 – 2011}
	{}
\end{cventries}

%%% SKILLS %%%
\cvsection{Competenze}
\begin{cventries}
	\cvskillentry
	{Linguaggi}
	{\begin{cventryparagraph}
		\item {Java, HTML, CSS, Javascript, C (Arduino), SQL.}
	\end{cventryparagraph}}
	
	\cvskillentry
	{Framework e librerie}
	{\begin{cventryparagraph}
		\item {Spring, REST API, Maven, Hibernate, MyBatis, Dojo, Highcharts, Owl Carousel.}
	\end{cventryparagraph}}
	
	\cvskillentry
	{Tools}
	{\begin{cventryparagraph}
		\item {GIT, SVN, Postman, Fiddler, Tomcat, WebSphere Portal, Eclipse, Netbeans, Visual Studio Code, AWS Cloudwatch, Office Suite, Trello.}
	\end{cventryparagraph}}
	
	\cvskillentry
	{Inglese}
	{\begin{cventryparagraph}
		\item {Buona conoscenza della lingua inglese acquisita durante l'università e i progetti di lavoro, durante i quali ho avuto l'opportunità di partecipare a riunioni con clienti internazionali.}
		\item {Un valido aiuto, soprattutto nel fare pratica diretta, l'ho avuto anche tramite il gaming online. Incontrando giocatori provenienti da molti paesi diversi, riuscire a comunicare e comprendersi è necessario per fare un buon gioco di squadra.}
	\end{cventryparagraph}}
	
	\cvskillentry
	{Teamwork and communication}
	{\begin{cventryparagraph}
		\item {Abilità di adeguamento a protocolli e metodologie utilizzate dal team, che siano agile o waterfall;}
		\item {capacità di fornire e ricevere feedback costruttivi;}
		\item {capacità di riconoscere le priorità e organizzare le attività di conseguenza.}
	\end{cventryparagraph}}
\end{cventries}

%%% CERTIFICATIONS %%%
\cvsection{Certificazioni}
\begin{cvhonors}
	\cvhonor
	{AWS Certified Cloud Practitioner}
	{}
	{AWS}
	{Ottobre 2022}
	\cvhonor
	{Oracle Certified Associate}
	{}
	{Oracle}
	{Dicembre 2021}
	\cvhonor
	{Corso Java Enterprise}
	{}
	{Consoft}
	{Novembre 2019}
	\cvhonor
	{Certificazione ECDL Core}
	{}
	{AICA}
	{Novembre 2009}
\end{cvhonors}

%%% PRIVACY %%%
\makecvfooter
{{\def\arraystretch{1.15}{\begin{tabular}{ l l }
		Note  & { Iscritto alle categorie protette, legge 68/99. } \\
		Trattamento dati personali  & { Autorizzo il trattamento dei dati personali contenuti nel mio curriculum vitae in base al D. Lgs. 196/2003, } \\
		& { integrato con le modifiche introdotte dal D. Lgs. 101/2018, e all’art. 13 del GDPR (Regolamento UE 2016/679) } \\
		& { ai fini della ricerca e selezione del personale. } \\
		\end{tabular}}}}
{}
{}

\
\end{document}